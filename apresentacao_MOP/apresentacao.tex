\documentclass{beamer} 
\usepackage[english]{babel}
%\usepackage[utf8x]{inputenc} % acento normal
\usepackage[utf8]{inputenc}
\usepackage{scalefnt}
\usepackage{color, colortbl, multirow}
\usepackage{epstopdf} %usar figura eps
\usepackage{verbatim} % para colocar o comment
%\usepackage[latin1]{inputenc}
\usepackage[algoruled,longend]{algorithm2e}



%\usepackage[applemac]{inputenc} % acento mac

\usepackage{graphicx} % para inserir figuras
%\graphicspath{figuras/} % para caminho da pasta com figuras

\usepackage{adjustbox}


\setbeamertemplate{navigation symbols}{} % remover barra de navega‹o
\setbeamertemplate{footline}[frame number] % numero de paginas


\usetheme {Frankfurt} %Berlin Malmoe Warsaw Rochester Darmstadt Madrid Antibes Bergen Berkley Boadilla Copenhagen Dresden Frankfurt 
%Goettingen Hannover Ilmenau JuanLesPins Luebeck Marburg Montpellier PaloAlto Pittsburgh Singapore Szeged boxes default CambridgeUS AnnArbor Rochester 

\usecolortheme{seahorse} %dolphin orchid whale
%albatross; beaver; beetle; crane; default; dolphin; dove; fly; lily; orchid; rose;seagull; seahorse; sidebartab; whale; wolverine

\definecolor{light_green}{RGB}{230,255,230} %cria a cor pra ser usada no background
\definecolor{tbGreen}{RGB}{166,210,166}

\setbeamercolor{background canvas}{bg=light_green} % aplica a cor no background
%\useinnertheme{circles} %define os numeradores com o padrão circles
%\setbeamercolor{block title}{fg=white,bg=green!40!black} %define os titulos dos blocos como verde
%\setbeamercolor{itemize item}{fg=green!40!black} %define os marcadores de ítem como verdes
%\setbeamercolor{itemize subitem}{fg=green!40!black} %define os marcadores de sub-ítem como verdes
%\setbeamercolor{item projected}{fg=white,bg=green!40!black} %define os numeradores como verdes

\setbeamercolor{section in toc shaded}{fg=black} %define os links no tableofcontents como preto (quando estão selecionados)
\setbeamercolor{section in toc}{fg=black} %define os links no tableofcontents como preto (quando não estão selecionados)

\setbeamercolor{structure}{bg=black,fg=green!50!black}
%\setbeamercolor{title}{fg=black, bg=green!40!black}
%\setbeamercolor{frametitle}{fg=black, bg=green!80!black}



\pgfdeclareimage[height=1cm]{c-bio}{figuras/c-bio.jpg}
%\logo{\pgfuseimage{c-bio}}
\pgfdeclareimage[height=1cm]{ufpr}{figuras/ufpr.png}
\logo{\pgfuseimage{ufpr} \hspace{276pt} \pgfuseimage{c-bio}}

%%===================== Capa =========================%%

\title{ \textbf{Otimização Multi-objetivo (MOP)} }

\author{Gian M. Fritsche %\\ \quad \\ \small{Advisor: Prof. Dr. Aurora T. 
% R. Pozo} 
}

\institute{Programa de Pós-Graduação em Informática\\ Universidade Federal do 
Paraná}

\begin{document} % inicio do documento

\date{9 de Março de 2015}

\frame{
	\titlepage
	%\center{\scriptsize{Orientadora: Prof.ª Aurora Trinidad Ramirez Pozo}}

}

\section{Otimização Multi-objetivo}
\subsection{O que é Otimização Multi-objetivo?}

\frame{
	\frametitle{O que é Otimização Multi-objetivo?}
	
	\begin{block}{}
		  Olá Mundo.
	\end{block}
}

\frame{
	\frametitle{Exemplo de problema Multi-objetivo}
	
	\begin{block}{}
		  Olá Mundo.
	\end{block}
}

\frame{
	\frametitle{O que é uma solução em um problema Multi-objetivo?}
	
	\begin{block}{}
		  Olá Mundo.
	\end{block}
}

\subsection{Exemplo de Figura}
\frame{
	\frametitle{Exemplo de Figura}

	\vspace{-4pt}
	\hspace{-20pt}
	\begin{minipage}{1.0\textwidth}
		\begin{figure}[ht]
		\centering
			\includegraphics[scale = 0.4]{figuras/ufpr.png}
			\caption{Exemplo de figura}
			\label{fig:exampleFig1}
		\end{figure}
	\end{minipage}
	
}


\frame{
	\frametitle{Exemplo de equação}
	
	    \begin{eqnarray}
		    \nonumber
	    	\label{eq:position}        
		    \vec{x}_i(t)=\vec{x}_i(t-1)+\vec{v}_i(t)
	    \end{eqnarray}
	
	%\begin{block}{MOPSO}
		%\begin{center}
			%\scriptsize{
			%\small
			\begin{eqnarray}
			\nonumber
%			\label{eq:velocity}        
			\vec{v}_i(t)=\underbrace{\omega \times \vec{v}_i(t-1)}_{\text{inertial}}+\underbrace{C_1 \times r_1 \times (\vec{x}_{b_i}-\vec{x}_i(t))}_{\text{cognitive}}+\underbrace{C_2 \times r_2 \times (\vec{x}_{g_i}-\vec{x}_i(t))}_{\text{social}}
			\end{eqnarray}
			%}
		%\end{center}
	%\end{block}
}


\frame{
	\frametitle{Exemplo de Tabela}
	
	\hspace{-20pt}
	\begin{minipage}{\textwidth}
		\begin{table}[t]
			\center 
			\normalsize
			%			\caption{$GD_p$ results} 
			\begin{tabular}{c|c|c|c|c}
				\hline
				Alg.&DTLZ1&DTLZ2&DTLZ3&DTLZ4\\ \hline
				I-Multi&41.77 (2.00)&\textbf{16.17 (1.00)}&44.07 (2.00)&\textbf{22.97 (1.00)}\\
				C-Multi&\textbf{19.23 (1.00)}&44.83 (2.00)&\textbf{16.93 (1.00)}&38.03 (2.00)\\ \hline
			\end{tabular}
			\begin{tabular}{c|c|c|c}
				\hline
				Alg.&DTLZ5&DTLZ6&DTLZ7 \\ \hline
				I-Multi&\textbf{33.30 (1.50)}&\textbf{22.27 (1.00)}&\textbf{26.07 (1.00)}\\
				C-Multi&\textbf{27.70 (1.50)}&38.73 (2.00)&34.93 (2.00)\\ \hline
			\end{tabular}
			%\label{tb:c-multi-gd}
		\end{table}
	\end{minipage}
}


%%====================================================%%

\frame{
	\frametitle{}
 
	\begin{block}{}
		\centering Obrigado! \\ \quad \\ \quad \\
		\centering Gian Mauricio Fritsche \\
		\centering \small gmfritsche@inf.ufpr.br
	\end{block}

}	

\end{document} 
