\documentclass{beamer} 
\usepackage[english]{babel}
%\usepackage[utf8x]{inputenc} % acento normal
\usepackage[utf8]{inputenc}
\usepackage{scalefnt}
\usepackage{color, colortbl, multirow}
\usepackage{epstopdf} %usar figura eps
\usepackage{verbatim} % para colocar o comment
%\usepackage[latin1]{inputenc}
\usepackage[algoruled,longend]{algorithm2e}



%\usepackage[applemac]{inputenc} % acento mac

\usepackage{graphicx} % para inserir figuras
%\graphicspath{figuras/} % para caminho da pasta com figuras

\usepackage{adjustbox}


\setbeamertemplate{navigation symbols}{} % remover barra de navega‹o
\setbeamertemplate{footline}[frame number] % numero de paginas


\usetheme {Frankfurt} %Berlin Malmoe Warsaw Rochester Darmstadt Madrid Antibes Bergen Berkley Boadilla Copenhagen Dresden Frankfurt 
%Goettingen Hannover Ilmenau JuanLesPins Luebeck Marburg Montpellier PaloAlto Pittsburgh Singapore Szeged boxes default CambridgeUS AnnArbor Rochester 

\usecolortheme{seahorse} %dolphin orchid whale
%albatross; beaver; beetle; crane; default; dolphin; dove; fly; lily; orchid; rose;seagull; seahorse; sidebartab; whale; wolverine

\definecolor{light_green}{RGB}{230,255,230} %cria a cor pra ser usada no background
\definecolor{tbGreen}{RGB}{166,210,166}

\setbeamercolor{background canvas}{bg=light_green} % aplica a cor no background
%\useinnertheme{circles} %define os numeradores com o padrão circles
%\setbeamercolor{block title}{fg=white,bg=green!40!black} %define os titulos dos blocos como verde
%\setbeamercolor{itemize item}{fg=green!40!black} %define os marcadores de ítem como verdes
%\setbeamercolor{itemize subitem}{fg=green!40!black} %define os marcadores de sub-ítem como verdes
%\setbeamercolor{item projected}{fg=white,bg=green!40!black} %define os numeradores como verdes

\setbeamercolor{section in toc shaded}{fg=black} %define os links no tableofcontents como preto (quando estão selecionados)
\setbeamercolor{section in toc}{fg=black} %define os links no tableofcontents como preto (quando não estão selecionados)

\setbeamercolor{structure}{bg=black,fg=green!50!black}
%\setbeamercolor{title}{fg=black, bg=green!40!black}
%\setbeamercolor{frametitle}{fg=black, bg=green!80!black}



\pgfdeclareimage[height=1cm]{c-bio}{figuras/c-bio.jpg}
%\logo{\pgfuseimage{c-bio}}
\pgfdeclareimage[height=1.2cm]{ufpr}{figuras/ufpr.jpg}
\logo{\pgfuseimage{ufpr} \hspace{276pt} \pgfuseimage{c-bio}}

%%===================== Capa =========================%%

\title{ \textbf{Particle Swarm Optimization algorithms for many-objective problems} }

\author{Olacir R. Castro Junior \\ \quad \\ \small{Advisor: Prof. Dr. Aurora T. R. Pozo} }

\institute{Computer Science's Department\\ Federal University of Paraná}

\begin{document} % inicio do documento

\date{February 05, 2015}

\frame{
	\titlepage
	%\center{\scriptsize{Orientadora: Prof.ª Aurora Trinidad Ramirez Pozo}}

}

%%===========================ROTEIRO=============================%%
\frame{
\frametitle{Schedule}

%\tableofcontents  %gera o sumario sozinho baseado nas \section e \subsection


%\begin{frame}{\contentsname}
   \hspace*{1cm}
   %\hspace*{30pt}
   %\vspace*{-20pt}
	\begin{minipage}[t][6cm][t]{\textwidth}
		\vspace{-30pt}
		\tableofcontents
	\end{minipage}
%\end{frame}

%\begin{block}{}
%\begin{enumerate}
%%	\item Introdução
%	\item Processo de Descoberta de Conhecimento
%	\item Classificação em Conjuntos de dados desbalanceados
%	\item Estudos Empíricos
%	\item Estudo de caso
%\end{enumerate}
%\end{block}
}

%%========================Introducao================================%%

\section{Introduction}
\subsection{Motivation}

\frame{
	\frametitle{Motivation}
	
	\begin{block}{}
		
		\begin{itemize}
			\item Multi-objective problems are successfully solved by existing algorithms.
			\item Traditional algorithms deteriorates the quality of its solutions as the number of objectives increases.
			\item Multi Objective Particle Swarm Optimization (MOPSO) is a good approach for optimizing multi-objective problems.
			\item The performance of MOPSO for many-objective problems can be enhanced.
		\end{itemize}	
	\end{block}
}

%%========================Introducao================================%%
\subsection{Objectives}
\frame{
	\frametitle{Objectives}
	
	\begin{block}{}
		
		\begin{itemize}
			\item Create a MOPSO-based algorithm less sensitive to the scale of the problems.
			\begin{itemize}
				\item Find good Estimation of Distribution Algorithms (EDAs) to hybridize with MOPSO.
				\item Find good hyper-heuristics to dynamically select components in MOPSO.
				\item Merge these approaches to create a hyper-heuristic guided EDA-MOPSO.
			\end{itemize}	
		\end{itemize}	
	\end{block}
}

%%========================Otimização================================%%

\section{Background}
\subsection{Optimization}
\frame{
	\frametitle{Optimization}
	
	\begin{block}{Mono-objective}
		
		\begin{itemize}
			\item Optimize $f(\vec{x})$, where $\vec{x}=(x_1,...,x_n)$ is an $n$-dimensional solution from an universe $\Omega$.	
		\end{itemize}	
	\end{block}
	
	\begin{block}{Multi-objective}
		\begin{itemize}
			\item Two or more objective functions
			\item Good compromise between the objectives
			\item Pareto optimality
			\begin{itemize}
				\item Non-dominated solutions
			\end{itemize}	
		\end{itemize}	
	\end{block}
}

%%========================Otimização Multi objetivo================================%%

\frame{
	\frametitle{Multi-Objetive problem example}
	
	\begin{adjustbox}{minipage=\linewidth-4pt,margin=0pt 5pt,bgcolor=white,frame=0pt}	
		%\vspace{-20pt}
		\begin{figure}[ht]
			\centering
			\includegraphics[scale = 0.9]{figuras/pareto_en.eps}
		\end{figure}
	\end{adjustbox}
	
}

%%========================Convergence================================%%

\frame{
	\frametitle{Convergence}
	
	\hspace{-20pt}
	\begin{adjustbox}{minipage=\linewidth-4pt,margin=0pt 5pt,frame=0pt}	
		\vspace{-8pt}
		\begin{figure}[ht]
			\centering
			\includegraphics[scale = 0.8]{figuras/convergence.eps}
		\end{figure}
	\end{adjustbox}
	
}

%%========================Diversity================================%%

\frame{
	\frametitle{Diversity}
	
	\hspace{-20pt}
	\begin{adjustbox}{minipage=\linewidth-4pt,margin=0pt 5pt,frame=0pt}	
		\vspace{-8pt}
		\begin{figure}[ht]
			\centering
			\includegraphics[scale = 0.8]{figuras/diversity.eps}
		\end{figure}
	\end{adjustbox}
	
}

%%========================Otimização com muitos objetivos================================%%
%\section{Many-Objective}
\frame{
	\frametitle{Many-Objective Optimization}

	\begin{block}{Many-Objectives}
		\begin{itemize}
			\item More than three objectives
			\item Real world problems
		\end{itemize}	
	\end{block}
	
	\begin{block}{Difficulties}
		\begin{itemize}
			\item Deterioration of the search ability
			\begin{itemize}
				\item Almost every solution becomes non-dominated
			\end{itemize}
			\item Great increase on the number of solutions to properly cover the entire front
		\end{itemize}	
	\end{block}
}

%%========================MOPSO================================%%

\subsection{MOPSO}
\frame{
	\frametitle{Multi-Objective Particle Swarm Optimization (MOPSO)}

	\vspace{-4pt}
	\hspace{-20pt}
	\begin{minipage}{1.0\textwidth}
		\begin{figure}[ht]
		\centering
			\includegraphics[scale = 0.4]{figuras/mopso.jpg}
			%\caption{Melhores configurações \cite{he}}
			\label{fig:exampleFig1}
		\end{figure}
	\end{minipage}
	
}

%%========================MOPSO================================%%

\frame{
	\frametitle{MOPSO}
	
	    \begin{eqnarray}
		    \nonumber
	    	\label{eq:position}        
		    \vec{x}_i(t)=\vec{x}_i(t-1)+\vec{v}_i(t)
	    \end{eqnarray}
	
	%\begin{block}{MOPSO}
		%\begin{center}
			%\scriptsize{
			%\small
			\begin{eqnarray}
			\nonumber
%			\label{eq:velocity}        
			\vec{v}_i(t)=\underbrace{\omega \times \vec{v}_i(t-1)}_{\text{inertial}}+\underbrace{C_1 \times r_1 \times (\vec{x}_{b_i}-\vec{x}_i(t))}_{\text{cognitive}}+\underbrace{C_2 \times r_2 \times (\vec{x}_{g_i}-\vec{x}_i(t))}_{\text{social}}
			\end{eqnarray}
			%}
		%\end{center}
	%\end{block}
}


%%========================MOPSO================================%%

\frame{
	\frametitle{MOPSO}
	
	\begin{block}{}
		\begin{itemize}
			\item Pareto Dominance
			\begin{itemize}
				\item Leader selection
			\end{itemize}
			\item External archive (Repository)
			\begin{itemize}
				\item Archiver
			\end{itemize}
			\item Speed-constrained Multi-objective PSO (SMPSO)
		\end{itemize}	
	\end{block}
}

%%========================Preliminary studies================================%%

\section{Preliminary Studies}
\frame{
	\frametitle{Preliminary Studies - seeking convergence and diversity}
	
	\begin{block}{Multi Swarm and Estimation of Distribution Algorithms (EDAs)}
		\begin{itemize}
			\item Brazilian Conference on Intelligent Systems (BRACIS) 2014.
		\end{itemize}	
	\end{block}
	\begin{block}{Hyper-Heuristics to explore leaders and archivers}
		\begin{itemize}
			\item IEEE Symposium Series on Computational Intelligence (SSCI) 2014.
			\item International Conference on Evolutionary Multi-Criterion Optimization (EMO) 2015.
		\end{itemize}	
	\end{block}
}

%%========================EDA================================%%

\subsection{EDA}
\frame{
	\frametitle{Estimation of Distribution Algorithms (EDA's)}
	
	\begin{block}{EDA's or Competent Evolutionary Algorithms}
		\begin{itemize}
			\item Exploit information obtained from the search
			\item Replace traditional operators by model sampling
			\item Characterized by its learning method
			\item real-coded Bayesian Optimization Algorithm (rBOA)
		\end{itemize}	
	\end{block}
	
}

%%========================EDA================================%%

\frame{
	\frametitle{Representation of a general EDA}
	
	\vspace{-4pt}
	%\hspace{-20pt}
	\begin{minipage}{1.0\textwidth}
		\begin{figure}[ht]
			\centering
			\includegraphics[scale = 0.52]{figuras/EDA.eps}
			\caption{Adapted from Karshenas et. al. 2012} % multi-objective estimation of distribution algorithms based on joint modeling of objectives and variables - pg 6
			%\label{fig:exampleFig1}
		\end{figure}
	\end{minipage}
}

%%========================C-multi================================%%

\frame{
	\frametitle{Competent Multi-swarm (C-Multi)}
	
	\begin{block}{}
		\begin{itemize}
			\item Diversity phase: SMPSO using MGA archiver
			\item Multi-swarm phase:
			\begin{itemize}
				\item Clustering
					\begin{itemize}
						\item Seed
						\item Search region
					\end{itemize}				
				\item rBOA and Ideal archiver
				\item Split iteration
			\end{itemize}
		\end{itemize}	
	\end{block}
	
}

%%========================C-Multi================================%%

\frame{
	\frametitle{Representation of C-Multi}
	
	\vspace{-4pt}
	%\hspace{-20pt}
	\begin{minipage}{1.0\textwidth}
		\begin{figure}[ht]
			\centering
			\includegraphics[scale = 0.6]{figuras/cmulti.eps}
			%\caption{Melhores configurações \cite{he}}
			\label{fig:exampleFig1}
		\end{figure}
	\end{minipage}
}

%%========================C-Multi================================%%

%\frame{
%	\frametitle{Competent Multi-swarm (C-Multi)}
%	
%	\vspace{-5pt}
%	\begin{minipage}{0.8\textwidth}
%		\scalebox{0.95}{ %0.95
%			\hspace{25pt}
%			\begin{algorithm}[H]
%				\label{alg:novo}
%				%\scriptsize{
%				\caption{C-Multi}
%				\tcp{Phase1: Diversity search}
%				
%				$F_b$=Run-MGA-SMPSO()
%				
%				\tcp{Phase 2: Multi-swarm search}
%				
%				\For(\tcp*[h]{Split iterations}){s=1 to SI}{
%					
%					$\vec{F}$=SplitFront($F_b$)
%					
%					\For(\tcp*[h]{Number of swarms}){k=1 to NS}{
%						
%						\For(\tcp*[h]{Iterations}){i=1 to It}{
%							
%							$P_k$=rBOA($F_k$)
%							
%							Truncate($P_k$)
%							
%							Evaluate($P_k$)
%							
%							$F_k$=IdealArchiving($P_k$)
%							
%						}
%						
%					}
%					
%					$F_b$=Non-dominated($\vec{F}$)
%					
%				}
%				
%				\textbf{return} $F_b$
%				%}
%				
%			\end{algorithm}
%			
%		}
%	\end{minipage}
%}

%%========================Empirical analysis================================%%

\frame{
	\frametitle{Empirical study}
	
	\begin{block}{Algorithms}
		\begin{itemize}
			\item I-Multi, C-Multi
		\end{itemize}
	\end{block}
	
	\begin{block}{Parameters}
		\begin{itemize}
			\item Entire DTLZ family of benchmark problems
			\item Iterations: 200 (Diversity: 100; multi-swarm: 100)
			\item Population: Diversity: 100; multi-swarm: 25 (per swarm)
			\item Number of split iterations: 5
			\item Number of swarms: 30
			\item Repository: 200
			\item Objectives: 3, 5, 10, 15, 20
			\item Leader threshold (learning): 0.3
			\item Bayesian Information Criterion $\lambda$ (learning): 0.5
		\end{itemize}
	\end{block}			
}

%%========================I-multi================================%%

\frame{
	\frametitle{Iterated Multi-swarm (I-Multi)}
	
	\begin{block}{}
		\begin{itemize}
			\item Diversity phase: SMPSO using MGA archiver
			\item Multi-swarm phase:
			\begin{itemize}
				\item Clustering
				\begin{itemize}
					\item Seed
					\item Search region
				\end{itemize}				
				\item SMPSO and Ideal archiver
				\item Split iteration
			\end{itemize}
		\end{itemize}	
	\end{block}
	
}

%%========================Empirical study================================%%

\frame{
	\frametitle{Empirical study}
	
	\begin{block}{Metrics}
		\begin{itemize}
			\item Modified Generational Distance ($GD_p$)\footnote{\label{foot}From Schütze et. al. 2012.}
			\item Modified Inverted Generational Distance ($IGD_p$)\textsuperscript{\ref{foot}}
			\item Kruskal–Wallis statistical test
			\begin{itemize}
				\item Significance level 5\%
				\item 30 independent runs
			\end{itemize}
		\end{itemize}
	\end{block}
	
}

%%========================Empirical study 3 obj================================%%

\frame{
	\frametitle{Kruskal–Wallis ranks of the $GD_p$ results - 3 Objectives}
	
	\hspace{-20pt}
	\begin{minipage}{\textwidth}
		\begin{table}[t]
			\center 
			\normalsize
%			\caption{$GD_p$ results} 
			\begin{tabular}{c|c|c|c|c}
				\hline
				Alg.&DTLZ1&DTLZ2&DTLZ3&DTLZ4\\ \hline
				I-Multi&\textbf{22.27 (1.00)}&\textbf{15.50 (1.00)}&\textbf{20.63 (1.00)}&\textbf{21.80 (1.00)}\\
				C-Multi&38.73 (2.00)&45.50 (2.00)&40.37 (2.00)&39.20 (2.00)\\ \hline
			\end{tabular}
			\begin{tabular}{c|c|c|c}
				\hline
				Alg.&DTLZ5&DTLZ6&DTLZ7 \\ \hline
				I-Multi&\textbf{22.73 (1.00)}&\textbf{16.53 (1.00)}&\textbf{17.43 (1.00)} \\
				C-Multi&38.27 (2.00)&44.47 (2.00)&43.57 (2.00)\\
			\end{tabular}
			%\label{tb:c-multi-gd}
		\end{table}
	\end{minipage}
}

%%========================Empirical study 20 obj================================%%

\frame{
	\frametitle{Kruskal–Wallis ranks of the $GD_p$ results - 20 Objectives}
	
	\hspace{-20pt}
	\begin{minipage}{\textwidth}
		\begin{table}[t]
			\center 
			\normalsize
			%			\caption{$GD_p$ results} 
			\begin{tabular}{c|c|c|c|c}
				\hline
				Alg.&DTLZ1&DTLZ2&DTLZ3&DTLZ4\\ \hline
				I-Multi&41.77 (2.00)&\textbf{16.17 (1.00)}&44.07 (2.00)&\textbf{22.97 (1.00)}\\
				C-Multi&\textbf{19.23 (1.00)}&44.83 (2.00)&\textbf{16.93 (1.00)}&38.03 (2.00)\\ \hline
			\end{tabular}
			\begin{tabular}{c|c|c|c}
				\hline
				Alg.&DTLZ5&DTLZ6&DTLZ7 \\ \hline
				I-Multi&\textbf{33.30 (1.50)}&\textbf{22.27 (1.00)}&\textbf{26.07 (1.00)}\\
				C-Multi&\textbf{27.70 (1.50)}&38.73 (2.00)&34.93 (2.00)\\ \hline
			\end{tabular}
			%\label{tb:c-multi-gd}
		\end{table}
	\end{minipage}
}

%%========================Empirical study 3 obj================================%%

\frame{
	\frametitle{Kruskal–Wallis ranks of the $IGD_p$ results - 3 Objectives}
	
	\hspace{-20pt}
	\begin{minipage}{\textwidth}
		\begin{table}[t]
			\center 
			\normalsize
			%			\caption{$GD_p$ results} 
			\begin{tabular}{c|c|c|c|c}
				\hline
				Alg.&DTLZ1&DTLZ2&DTLZ3&DTLZ4\\ \hline
				I-Multi&\textbf{21.13 (1.00)}&\textbf{15.50 (1.00)}&\textbf{22.18 (1.00)}&\textbf{19.90 (1.00)}\\
				C-Multi&39.87 (2.00)&45.50 (2.00)&38.82 (2.00)&41.10 (2.00)\\ \hline
			\end{tabular}
			\begin{tabular}{c|c|c|c}
				\hline
				Alg.&DTLZ5&DTLZ6&DTLZ7 \\ \hline
				I-Multi&\textbf{15.53 (1.00)}&\textbf{26.70 (1.50)}&\textbf{20.77 (1.00)}\\
				C-Multi&45.47 (2.00)&\textbf{34.30 (1.50)}&40.23 (2.00)\\ \hline
			\end{tabular}
			%\label{tb:c-multi-gd}
		\end{table}
	\end{minipage}
}

%%========================Empirical study 20 obj================================%%

\frame{
	\frametitle{Kruskal–Wallis ranks of the $IGD_p$ results - 20 Objectives}
	
	\hspace{-20pt}
	\begin{minipage}{\textwidth}
		\begin{table}[t]
			\center 
			\normalsize
			%			\caption{$GD_p$ results} 
			\begin{tabular}{c|c|c|c|c}
				\hline
				Alg.&DTLZ1&DTLZ2&DTLZ3&DTLZ4\\ \hline
				I-Multi&\textbf{26.12 (1.50)}&\textbf{15.50 (1.00)}&\textbf{26.03 (1.00)}&\textbf{25.97 (1.00)}\\
				C-Multi&\textbf{34.88 (1.50)}&45.50 (2.00)&34.97 (2.00)&35.03 (2.00)\\ \hline
			\end{tabular}
			\begin{tabular}{c|c|c|c}
				\hline
				Alg.&DTLZ5&DTLZ6&DTLZ7 \\ \hline
				I-Multi&\textbf{22.67 (1.00)}&\textbf{26.17 (1.50)}&\textbf{26.20 (1.50)}\\
				C-Multi&38.33 (2.00)&\textbf{34.83 (1.50)}&\textbf{34.80 (1.50)}\\ \hline
			\end{tabular}
			%\label{tb:c-multi-gd}
		\end{table}
	\end{minipage}
}


%%========================Discussion================================%%

\frame{
	\frametitle{Discussion}
	
	\begin{block}{}
		\begin{itemize}
			\item I-Multi - Better overall performance
			\item C-Multi - Good performance in problems presenting convergence challenge (DTLZ1,DTLZ3)
			\item Solutions sampled from model are closer to the better, enhancing convergence
			\item These results are promising, encouraging further development on this area.
			\begin{itemize}
				\item Different EDAs - CMA-ES
				\item Calibration of the parameters.
			\end{itemize}
		\end{itemize}
	\end{block}		
	
%	\begin{block}{Future works}
%		\begin{itemize}
%			\item Better calibration of the parameters
%			\item Using other learning algorithms
%		\end{itemize}
%	\end{block}		
}

%%========================Transicao================================%%

\frame{
	\frametitle{Seeking for convergence and diversity}
	
	\begin{block}{}
		\begin{itemize}
			\item Different approach
			\item Archiving, leader selection
			\begin{itemize}
				\item Improve results
				\item Problem dependent
			\end{itemize}
		\end{itemize}
	\end{block}		
	
	%	\begin{block}{Future works}
	%		\begin{itemize}
	%			\item Better calibration of the parameters
	%			\item Using other learning algorithms
	%		\end{itemize}
	%	\end{block}		
}

%%========================Hyper-heuristics================================%%
\subsection{Hyper-heuristics}
\frame{
	\frametitle{Hyper-heuristics}
	
	\begin{block}{}
		\begin{itemize}
			\item Choosing parameters or algorithms to solve a problem is hard.
			\item Combine algorithms with different strengths and weaknesses.
			\item Hyper-heuristic as a high-level methodology.
				\begin{itemize}
					\item Heuristic selection.
					\item Move acceptance.
				\end{itemize}
		\end{itemize}
	\end{block}			
}

%%========================Hyper-heuristic================================%%
\frame{
	\frametitle{Hyper-heuristic}
	
	\vspace{-4pt}
	\hspace{-30pt}
	\begin{minipage}{1.0\textwidth}
		\begin{figure}[ht]
			\centering
			\colorbox{white}{\includegraphics[width=1.15\textwidth]{figuras/hyper-heuristic.eps}}
			\caption{Adapted from Glover and Kochenberger, 2003} %handbook of metaheuristics page 480
			\label{fig:hyper-heuristic}
		\end{figure}
	\end{minipage}
	
}

%%========================Hyper-heuristics================================%%
\frame{
	\frametitle{H-MOPSO}
	
	\begin{block}{}
		\begin{itemize}
			\item Different archivers in the literature
				\begin{itemize}
					\item Crowding Distance (CD), Multilevel Grid Archiving (MGA), Ideal, ...
				\end{itemize}
			\item Different leader selection methods
					\begin{itemize}
						\item Crowding Distance (CD), Weighted Sum (WSum), Sigma, ...
					\end{itemize}
			\item Metric instead of direct fitness value: $R_2$
			\item Heuristic selection: Roulette wheel
			\item Move acceptance: Improving or Equal (IE) 
		\end{itemize}
	\end{block}			
}

%%========================Hyper-heuristics================================%%
\frame{
	\frametitle{H-MOPSO}
	
	\begin{block}{}
		\begin{itemize}
			\item Run SMPSO (CD leader) until the repository is full
			\item At each iteration:			
			\begin{itemize}
				\item Select a low-level heuristic using the roulette.
				\item Update the particles and the repository.
				\item Update the probabilities in the roulette according to the $R_2$ performance.
				\item Restore a copy of the repository if the result became worse.
			\end{itemize}
		\end{itemize}
	\end{block}			
}

%%%========================H-MOPSO================================%%
%\frame{
%	\frametitle{Representation of H-MOPSO}
%	
%	%\vspace{-4pt}
%	\begin{block}{}
%		\begin{itemize}
%			\item Roulette wheel 
%			\item $R_2$ indicator
%			\item Improving or Equal (IE)
%		\end{itemize}	
%	\end{block}
%	
%	\hspace{-30pt}
%	\begin{minipage}{1.0\textwidth}
%		\begin{figure}[ht]
%			\centering
%			\colorbox{white}{\includegraphics[width=1.15\textwidth]{figuras/h-mopso.eps}}
%		\end{figure}
%	\end{minipage}
%}

%%%========================H-MOPSO================================%%
%
%
%\frame{
%	\frametitle{H-MOPSO}
%	
%	\vspace{-5pt}
%	\hspace{30pt}
%	\begin{minipage}{0.8\textwidth}
%		\scalebox{0.49}{ %0.95
%			\hspace{25pt}
%				\begin{algorithm}[H]
%					\label{alg:novo}
%					%\scriptsize{
%					%\small
%					\caption{H-MOPSO}
%					
%					initialize(particles)
%					
%					repository=initializeRepository(particles)
%					
%					roulette=initializeRoulette();
%					
%					$t=0$
%					
%					\While{$t < tmax$}{
%						
%						\If{repository is full}{
%							
%							selectLeaderArchiver(roulette)
%							
%						}\Else{
%						selectLeaderArchiver(CD-CD)
%					}
%					
%					\For{each particle in the repository}{
%						
%						selectGlobalLeader(particle, repository)
%						
%						updatePosition(particle)
%						
%						mutation(particle)
%						
%						evaluation(particle)
%						
%						updatePersonalLeader(particle)
%						
%					}
%					
%					\If{repository is full}{
%						repositoryPrevious=repository
%						
%						repository=updateRepository(particles)
%						
%						$R_2$=calculate$R_2$(repository)
%						
%						$R_2$Ant=calculate$R_2$(repositoryPrevious)
%						
%						roulette=updateRoulette(roulette, $R_2$Ant, $R_2$)
%						
%						\If{$R_2$Ant $<$ $R_2$}{
%							
%							repository=repositoryPrevious
%							
%						}
%					}\Else{
%					repository=updateRepository(particles)
%				}
%				
%				
%				
%				$t++$
%				
%			}
%			
%			\textbf{return} repository
%			%}
%			
%		\end{algorithm}
%			
%		}
%	\end{minipage}
%}

%%========================Empirical analysis================================%%

\frame{
	\frametitle{Empirical study}
	
	\begin{block}{Algorithms}
		\begin{itemize}
			\item H-MOPSO and the low-level heuristics used separately.
		\end{itemize}
	\end{block}
	
	\begin{block}{Parameters}
		\begin{itemize}
			\item Entire DTLZ family of benchmark problems
			\item Iterations: 100
			\item Population: 100
			\item Repository: 100
			\item Objectives: 2, 3, 5, 10, 15, 20
			\item Minimum probability: 0.5\%
		\end{itemize}
	\end{block}	
}

%%========================Empirical study================================%%

\frame{
	\frametitle{Empirical study}
	
	\begin{block}{Metric}
		\begin{itemize}
			\item $R_2$
			\item Kruskal–Wallis (per problem) Friedman (overall) statistical tests
			\begin{itemize}
				\item Significance level 5\%
				\item 30 independent runs
			\end{itemize}
		\end{itemize}
	\end{block}
}

%%========================Empirical study overall================================%%

\frame{
	\frametitle{Friedman overall ranks of the $R_2$ results}
	
	\hspace{-20pt}
	\begin{minipage}{\textwidth}
		\begin{table}[t]
			\center 
			\normalsize
			%			\caption{$GD_p$ results} 
			\begin{tabular}{c|c}
				\hline
				H-MOPSO & \textbf{61.0 (1.0)} \\
				SMPSO & 171.0 (4.5) \\
				CD-Ideal & 302.0 (7.5) \\
				CD-MGA & 226.0 (5.0) \\
				NWSum-CD & 198.0 (4.5) \\
				NWSum-Ideal & 255.0 (6.0) \\
				NWSum-MGA & 227.0 (5.0) \\
				Sigma-CD & 211.0 (4.5) \\
				Sigma-Ideal & 324.0 (8.5) \\
				Sigma-MGA & 335.0 (8.5) \\ \hline
			\end{tabular}
			%\label{tb:c-multi-gd}
		\end{table}
	\end{minipage}
}

%%========================Empirical analysis================================%%

\frame{
	\frametitle{Empirical study}
	
	\begin{block}{Algorithms}
		\begin{itemize}
			\item H-MOPSO and MOEA/D-DRA.
		\end{itemize}
	\end{block}
	
	\begin{block}{Parameters}
		\begin{itemize}
			\item Entire DTLZ family of benchmark problems
			\item Iterations: 100
			\item Population: 100
			\item Repository: 100
			\item Objectives: 2, 3, 5, 8, 9.
			\item Minimum probability: 0.5\%
		\end{itemize}
	\end{block}	
}

%%========================Empirical study================================%%

\frame{
	\frametitle{Empirical study}
	
	\begin{block}{Metric}
		\begin{itemize}
			\item $IGD_p$, Hypervolume
			\item Kruskal–Wallis
			\begin{itemize}
				\item Significance level 5\%
				\item 30 independent runs
			\end{itemize}
		\end{itemize}
	\end{block}
}

%%========================Empirical study 2 obj================================%%

\frame{
	\frametitle{Kruskal–Wallis ranks of the $IGD_p$ results - 2 Objectives}
	
	\hspace{-25pt}
	\begin{minipage}{\textwidth}
		\begin{table}[t]
			\center 
			\small
			%			\caption{$GD_p$ results} 
			\begin{tabular}{c|c|c|c|c}
				\hline
				Alg.&DTLZ1&DTLZ2&DTLZ3&DTLZ4\\ \hline
				H-MOPSO&\textbf{16.90 (1.00)}&\textbf{15.50 (1.00)}&\textbf{15.50 (1.00)}&\textbf{17.70 (1.00)}\\
				MOEA/D-DRA&44.10 (2.00)&45.50 (2.00)&45.50 (2.00)&43.30 (2.00)\\ \hline
			\end{tabular}
			\begin{tabular}{c|c|c|c}
				\hline
				Alg.&DTLZ5&DTLZ6&DTLZ7 \\ \hline
				H-MOPSO&\textbf{15.50 (1.00)}&\textbf{16.50 (1.00)}&\textbf{15.50 (1.00)}\\
				MOEA/D-DRA&45.50 (2.00)&44.50 (2.00)&45.50 (2.00)\\ \hline
			\end{tabular}
			%\label{tb:c-multi-gd}
		\end{table}
	\end{minipage}
}

%%========================Empirical study 9 obj================================%%

\frame{
	\frametitle{Kruskal–Wallis ranks of the $IGD_p$ results - 9 Objectives}
	
	\hspace{-25pt}
	\begin{minipage}{\textwidth}
		\begin{table}[t]
			\center 
			\small
			%			\caption{$GD_p$ results} 
			\begin{tabular}{c|c|c|c|c}
				\hline
				Alg.&DTLZ1&DTLZ2&DTLZ3&DTLZ4\\ \hline
				H-MOPSO&\textbf{17.37 (1.00)}&36.37 (2.00)&\textbf{15.50 (1.00)}&\textbf{16.23 (1.00)}\\
				MOEA/D-DRA&43.63 (2.00)&\textbf{24.63 (1.00)}&45.50 (2.00)&44.77 (2.00)\\ \hline
			\end{tabular}
			\begin{tabular}{c|c|c|c}
				\hline
				Alg.&DTLZ5&DTLZ6&DTLZ7 \\ \hline
				H-MOPSO&45.50 (2.00)&40.67 (2.00)&44.57 (2.00)\\
				MOEA/D-DRA&\textbf{15.50 (1.00)}&\textbf{20.33 (1.00)}&\textbf{16.43 (1.00)}\\ \hline
			\end{tabular}
			%\label{tb:c-multi-gd}
		\end{table}
	\end{minipage}
}

%%========================Empirical study 2 obj================================%%

\frame{
	\frametitle{Kruskal–Wallis ranks of the Hypervolume results - 2 Objectives}
	
	\hspace{-25pt}
	\begin{minipage}{\textwidth}
		\begin{table}[t]
			\center 
			\small
			%			\caption{$GD_p$ results} 
			\begin{tabular}{c|c|c|c|c}
				\hline
				Alg.&DTLZ1&DTLZ2&DTLZ3&DTLZ4\\ \hline
				H-MOPSO&\textbf{15.80 (1.00)}&\textbf{15.53 (1.00)}&\textbf{15.50 (1.00)}&\textbf{16.73 (1.00)}\\
				MOEA/D-DRA&45.20 (2.00)&45.47 (2.00)&45.50 (2.00)&44.27 (2.00)\\ \hline
			\end{tabular}
			\begin{tabular}{c|c|c|c}
				\hline
				Alg.&DTLZ5&DTLZ6&DTLZ7 \\ \hline
				H-MOPSO&\textbf{15.50 (1.00)}&\textbf{16.50 (1.00)}&\textbf{15.50 (1.00)}\\
				MOEA/D-DRA&45.50 (2.00)&44.50 (2.00)&45.50 (2.00)\\ \hline
			\end{tabular}
			%\label{tb:c-multi-gd}
		\end{table}
	\end{minipage}
}

%%========================Empirical study 9 obj================================%%

\frame{
	\frametitle{Kruskal–Wallis ranks of the Hypervolume results - 9 Objectives}
	
	\hspace{-25pt}
	\begin{minipage}{\textwidth}
		\begin{table}[t]
			\center 
			\small
			%			\caption{$GD_p$ results} 
			\begin{tabular}{c|c|c|c|c}
				\hline
				Alg.&DTLZ1&DTLZ2&DTLZ3&DTLZ4\\ \hline
				H-MOPSO&\textbf{15.60 (1.00)}&44.73 (2.00)&\textbf{15.50 (1.00)}&38.03 (2.00)\\
				MOEA/D-DRA&45.40 (2.00)&\textbf{16.27 (1.00)}&45.50 (2.00)&\textbf{22.97 (1.00)}\\ \hline
			\end{tabular}
			\begin{tabular}{c|c|c|c}
				\hline
				Alg.&DTLZ5&DTLZ6&DTLZ7 \\ \hline
				H-MOPSO&\textbf{15.50 (1.00)}&\textbf{17.57 (1.00)}&\textbf{15.50 (1.00)}\\
				MOEA/D-DRA&45.50 (2.00)&43.43 (2.00)&45.50 (2.00)\\ \hline
			\end{tabular}
			%\label{tb:c-multi-gd}
		\end{table}
	\end{minipage}
}

%%========================Empirical analysis================================%%

\frame{
	\frametitle{Probabilities of choosing a low-level heuristic over time}
	
	\vspace{-4pt}
	\hspace{-12pt}
	\begin{minipage}{0.9\textwidth}
		\begin{figure}[ht]
			\centering
			\colorbox{white}{\includegraphics[width=1.15\textwidth]{figuras/plot-DTLZ5-2-Probability.eps}}
		\end{figure}
	\end{minipage}
}

%%========================Empirical analysis================================%%

\frame{
	\frametitle{Probabilities of choosing a low-level heuristic over time}
	
	\vspace{-4pt}
	\hspace{-12pt}
	\begin{minipage}{0.9\textwidth}
		\begin{figure}[ht]
			\centering
			\colorbox{white}{\includegraphics[width=1.15\textwidth]{figuras/plot-DTLZ5-20-Probability.eps}}
		\end{figure}
	\end{minipage}
}


%%========================Discussion================================%%

\frame{
	\frametitle{Discussion}
	
	\begin{block}{}
		\begin{itemize}
			\item H-MOPSO guide the search through its low-level heuristics.
			\item There is no single low-level heuristic that excels in all the problems.
			\item H-MOPSO is competitive with MOEA/D-DRA.
			\item Large number of iterations for the hyper-heuristic to work.
			\item High usage of sub-optimal heuristics in some circumstances.
			\item These results are promising, encouraging further development on this area.
			\begin{itemize}
				\item Different Hyper-Heuristics - FRRMAB
				\item Different metrics.
			\end{itemize}
		\end{itemize}
	\end{block}		
	
%	\begin{block}{Future works}
%		\begin{itemize}
%			\item Different performance indicators
%			\item Control of more parameters or components
%			\item Different selection hyper-heuristics
%		\end{itemize}
%	\end{block}		
}

%%========================Proposal================================%%

\section{Proposal}
\frame{
	\frametitle{Proposal}
	
	\begin{block}{Future challenges}
		\begin{itemize}
			\item Study EDAs to hybridize with MOPSOs.
				\begin{itemize}
					\item Sandwich period, co-oriented by Prof. Dr. Roberto Santana
				\end{itemize}
			\item Develop hyper-heuristic based MOPSOs.
				\begin{itemize}
					\item In cooperation with MSc. student Gian Fritsche.
				\end{itemize}
			\item Merge the two approaches to create a hyper-heuristic guided EDA-MOPSO.
		\end{itemize}	
	\end{block}
}

%%========================Schedule================================%%
%\subsection{Schedule}
\frame{
	\frametitle{Schedule}
	
	\hspace{-28pt}
	\begin{minipage}{\textwidth}
\begin{table}[htbp]
	\tiny
	\centering
	\begin{tabular}{|l|c|c|c|c|c|c|c|c|c|c|c|c|}
		\hline
		\multicolumn{ 13}{|c|}{2015} \\ \hline
		& Jan & Feb & Mar & Apr & May & Jun & Jul & Aug & Sep & Oct & Nov & Dec \\ \hline
		Preparation to sandwich & \multicolumn{1}{l|}{} &  & X &  &  &  &  &  &  &  &  &  \\ \hline
		Sandwich period & \multicolumn{1}{l|}{} & \multicolumn{1}{l|}{} &  & \cellcolor{tbGreen}X & \cellcolor{tbGreen}X & \cellcolor{tbGreen}X & \cellcolor{tbGreen}X & \cellcolor{tbGreen}X & \cellcolor{tbGreen}X & \cellcolor{tbGreen}X & \cellcolor{tbGreen}X & \cellcolor{tbGreen}X \\ \hline
		EDAs & \multicolumn{1}{l|}{} & \multicolumn{1}{l|}{} &  & \cellcolor{tbGreen}X & \cellcolor{tbGreen}X & \cellcolor{tbGreen}X & \cellcolor{tbGreen}X & \cellcolor{tbGreen}X & \cellcolor{tbGreen}X & \cellcolor{tbGreen}X & \cellcolor{tbGreen}X & \cellcolor{tbGreen}X \\ \hline
		Hyper-heuristics & \multicolumn{1}{l|}{} & \multicolumn{1}{l|}{} &  & \cellcolor{tbGreen}X & \cellcolor{tbGreen}X & \cellcolor{tbGreen}X & \cellcolor{tbGreen}X & \cellcolor{tbGreen}X & \cellcolor{tbGreen}X & \cellcolor{tbGreen}X & \cellcolor{tbGreen}X & \cellcolor{tbGreen}X \\ \hline
		Paper writing & \multicolumn{1}{l|}{} & \multicolumn{1}{l|}{} &  &  &  & \cellcolor{tbGreen}X & \cellcolor{tbGreen}X & \cellcolor{tbGreen}X & \cellcolor{tbGreen}X & \cellcolor{tbGreen}X & \cellcolor{tbGreen}X & \cellcolor{tbGreen}X \\ \hline
		\multicolumn{ 13}{|c|}{2016} \\ \hline
		& Jan & Feb & Mar & Apr & May & Jun & Jul & Aug & Sep & Oct & Nov & Dec \\ \hline
		Preparation from sandwich &  &  &  & X &  &  &  &  &  &  &  &  \\ \hline
		Sandwich period & \cellcolor{tbGreen}X & \cellcolor{tbGreen}X & \cellcolor{tbGreen}X &  &  &  &  &  &  &  &  &  \\ \hline
		EDAs & \cellcolor{tbGreen}X & \cellcolor{tbGreen}X & \cellcolor{tbGreen}X &  &  &  &  &  &  &  &  &  \\ \hline
		Hyper-heuristics & \cellcolor{tbGreen}X & \cellcolor{tbGreen}X & \cellcolor{tbGreen}X &  &  &  &  &  &  &  &  &  \\ \hline
		Final algorithm &  &  &  &  & X & X & X & X & X & X &  &  \\ \hline
		Paper writing & \cellcolor{tbGreen}X & \cellcolor{tbGreen}X & \cellcolor{tbGreen}X &  &  & X & X & X & X & X &  &  \\ \hline
		Elaboration of thesis &  &  &  &  &  &  &  &  &  &  & X & X \\ \hline
		\multicolumn{ 13}{|c|}{2017} \\ \hline
		& Jan & Feb & Mar & Apr & May & Jun & Jul & Aug & Sep & Oct & Nov & Dec \\ \hline
		Elaboration of thesis & X &  &  & \multicolumn{1}{l|}{} & \multicolumn{1}{l|}{} & \multicolumn{1}{l|}{} & \multicolumn{1}{l|}{} & \multicolumn{1}{l|}{} & \multicolumn{1}{l|}{} & \multicolumn{1}{l|}{} & \multicolumn{1}{l|}{} & \multicolumn{1}{l|}{} \\ \hline
		Prepare thesis presentation & \multicolumn{1}{l|}{} & X &  & \multicolumn{1}{l|}{} & \multicolumn{1}{l|}{} & \multicolumn{1}{l|}{} & \multicolumn{1}{l|}{} & \multicolumn{1}{l|}{} & \multicolumn{1}{l|}{} & \multicolumn{1}{l|}{} & \multicolumn{1}{l|}{} & \multicolumn{1}{l|}{} \\ \hline
		Thesis presentation & \multicolumn{1}{l|}{} &  & X & \multicolumn{1}{l|}{} & \multicolumn{1}{l|}{} & \multicolumn{1}{l|}{} & \multicolumn{1}{l|}{} & \multicolumn{1}{l|}{} & \multicolumn{1}{l|}{} & \multicolumn{1}{l|}{} & \multicolumn{1}{l|}{} & \multicolumn{1}{l|}{} \\ \hline
	\end{tabular}
\end{table}
	\end{minipage}
}

%%========================Conclusions================================%%
\section{Conclusion}
\frame{
	\frametitle{Conclusions}
	
	\begin{block}{}
		\begin{itemize}
			\item Increased number of objectives makes it harder to optimize a problem.
			\item Different approaches can enhance the performance of MOPSOs in such problems.
			\item Due to promising results in preliminary works, we will keep focusing in EDAs and hyper-heuristics.
		\end{itemize}
	\end{block}		
	
	%	\begin{block}{Future works}
	%		\begin{itemize}
	%			\item Different performance indicators
	%			\item Control of more parameters or components
	%			\item Different selection hyper-heuristics
	%		\end{itemize}
	%	\end{block}		
}

%%====================================================%%

\frame{
	\frametitle{}
 
	\begin{block}{}
		\centering Thanks! \\ \quad \\ \quad \\
		\centering Olacir R. Castro Junior \\
		\centering \small olacirjr@gmail.com
	\end{block}

	
	
}	

\end{document} 
